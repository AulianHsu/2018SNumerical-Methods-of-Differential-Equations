\documentclass{article}%book,report,letter

\usepackage{ctex}
\usepackage{fontspec}
%\usepackage{color}
%\usepackage{graphicx} %use graph format
%\usepackage{subfigure}
%\usepackage{epstopdf} %eps图片
\usepackage{amsmath}  %字体加粗
%\usepackage{math}
\usepackage{amsthm}
\usepackage{amssymb} %因为所以符号
%\usepackage{caption}
%\captionsetup[table]{labelsep=space}
\usepackage{float}%图片位置

%自定义命令
\newcommand*{\myTestTimes}{1\xspace}
%\typein[\myTestTimes]{这是第几次测试?}
\newcommand*{\myName}{桑明达\xspace}
\newcommand*{\myNumber}{15300180062\xspace}
\newcommand*{\myHomeworkNumber}{第十三周作业\xspace}
\newcommand*{\myArticleName}{微分方程数值解法\xspace}

\newcommand*{\myseries}[2][n]{\ensuremath{#2_1,#2_2,\dots,#2_{#1}}}


%制作页眉页脚
\usepackage{fancyhdr}
\pagestyle{fancy}
\lhead{\myHomeworkNumber}
\chead{\myArticleName}
\rhead{\myName \myNumber}
\lfoot{}
\cfoot{\thepage}
\rfoot{}
\renewcommand{\headrulewidth}{0.4pt}
\renewcommand{\footrulewidth}{0.4pt}

%标题
\title{\heiti \myArticleName \\ [2ex] \begin{large} \myHomeworkNumber \end{large}}
\author{\kaishu \myName \myNumber}
\date{\today}

% 正文区
\begin{document}
\maketitle
%\newpage
\section{P207 DuFort-Frankel格式截断误差}
\begin{proof}
	\begin{align*}
		R^n_i=&\frac{u^{n+1}_i-u^{n+1}}{2\tau}-a\frac{u^{n}_{i+1}-(u^{n+1}_i+u^{n-1}_i)+u^{n}_{i-1}}{h^2}-(u_t \big|^n_i-au_{xx} \big|^n_i) \\
		\text{分别定义}&R_t \big|^n_i,R_x\big|^n_i \\
		R_t \big|^n_i=&\frac{u^{n+1}_i-u^{n+1}}{2\tau}-u_t \big|^n_i \\
		=&\frac{\tau^2}{6}u_{tt}\big|^n_i+O(\tau^3) \\
		=&O(\tau^2) \\
		R_x\big|^n_i=&-a\frac{u^{n}_{i+1}-(u^{n+1}_i+u^{n-1}_i)+u^{n}_{i-1}}{h^2}+au_{xx} \big|^n_i \\
		=&-\frac{a}{h^2}\left ((u^{n}_{i+1}+u^{n}_{i-1})-(u^{n+1}_i+u^{n-1}_i)  \right )+au_{xx} \big|^n_i \\
		=&-\frac{a}{h^2}\left ((2u^{n}_{i}+h^2u_{xx}\big|^n_i+O(h^4))-(2u^{n}_{i}+\tau^2u_{tt}\big|^n_i+O(\tau^4))  \right )+au_{xx} \big|^n_i \\
		=&-\frac{a}{h^2}\left (O(h^4)-\tau^2u_{tt}\big|^n_i-O(\tau^4)  \right ) \\
		=&O(h^2)+O(\tau^2h^{-2}) \\
		\text{所以,}R^n_i=&O(\tau^2+h^2)+O(\tau^2h^{-2}) \\
	\end{align*}
\end{proof}
\end{document}
