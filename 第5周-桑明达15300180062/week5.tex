
%导言区
\documentclass{article}%book,report,letter

\usepackage{ctex}
\usepackage{fontspec}
%\usepackage{color}
%\usepackage{graphicx} %use graph format
%\usepackage{subfigure}
%\usepackage{epstopdf} %eps图片
\usepackage{amsmath}  %字体加粗
%\usepackage{math}
\usepackage{amsthm}
%制作页眉页脚
\usepackage{fancyhdr}  
\pagestyle{fancy}  
\lhead{第五周作业}  
\chead{微分方程数值解法}  
\rhead{桑明达 15300180062}  
\lfoot{}  
\cfoot{\thepage}  
\rfoot{}  
\renewcommand{\headrulewidth}{0.4pt}  
\renewcommand{\footrulewidth}{0.4pt} 

%标题
\author{names}
\title{\heiti 微分方程数值解法\\ [2ex] \begin{large} 第五周作业 \end{large}}
\author{\kaishu 桑明达 15300180062}
\date{\today}

% 正文区
\begin{document}
\maketitle

%\newpage

\section{P84 1 证明引理2.3.5}

\begin{proof}
	$ \phi \left ( t,u;\Delta t \right ) $满足Lipschitz条件,即
\begin{align*}
	\left |\phi \left ( t_{n},u_{n}^{\epsilon };\Delta t \right )-\phi \left ( t_{n}^{},u_{n};\Delta t \right )  \right | & \leq  L \left | u_{n}^{\epsilon }-u_{n} \right |  \\
	\left | u_{n+1}^{\epsilon }-u_{n+1} \right |& =   \left | u_{n}^{\epsilon }+\Delta t \phi \left ( t_{n},u_{n}^{\epsilon };\Delta t \right )-u_{n} -\Delta t \phi \left ( t_{n},u_{n};\Delta t \right ) \right | \\
	& \leq \left | \left (u_{n}^{\epsilon }-u_{n}  \right )\left ( 1+\Delta t L \right ) \right | \\
	& \leq \left ( 1+\Delta t L \right )^{n+1} \left | \left (u_{0}^{\epsilon }-u_{0}  \right ) \right | \\
	\text{对于$ 0 < t \leq T = N \Delta t $,有} \\
	\left | u_{n+1}^{\epsilon }-u_{n+1} \right |& \leq e^{LT}\epsilon \\
	\text{所以单步方法稳定。} \\
\end{align*}
\end{proof}

\section{P85 2 证明定理2.3.6}

\begin{proof}
	$ \epsilon _{n+1}=u(t_{n+1})-u_{n+1} $,代入隐式Euler格式,有
	\begin{align*}
	\left |\epsilon _{n+1}  \right | = &  \left| u\left (t_{n+1}  \right ) - u_{n+1} \right| \\
	= & \left| u\left (t_{n+1}  \right ) - u_n - \Delta t \phi \left ( t_{n},u_{n};\Delta t \right )  \right| \\
	\leq & \left| u\left (t_{n+1}  \right ) - u \left (t_n   \right )- \Delta t \phi \left ( t_{n},u \left (t_n   \right );\Delta t \right ) \right| + \left| u\left (t_{n}  \right ) - u_n  \right| \\
	& + \left| \Delta t \phi \left ( t_{n},u\left (t_n   \right );\Delta t \right ) - \Delta t \phi \left ( t_{n},u_{n};\Delta t \right ) \right| \\
	= & \left | R_{n+1} \right | +\left | \epsilon _{n} \right | + \Delta t L \left | \epsilon _{n} \right | \\
	\leq & C_{R}\Delta t^{p+1}  + \left (1+\Delta t L  \right ) \left | \epsilon _{n} \right | \\
	\leq &  C_{R}\Delta t^{p+1} \frac{\left ( 1+\Delta tL \right )^{n+1} - 1}{ \Delta t L}+\left ( 1+\Delta tL \right )^{n+1}\left |\epsilon _{0} \right | \\
	\leq & C_{R}\Delta t^{p} \frac{e^{L\left (T-t_{0}  \right )} }{  L}+e^{L\left (T-t_{0}  \right )}\left |\epsilon _{0}  \right |	
	\end{align*}
\end{proof}

\end{document}