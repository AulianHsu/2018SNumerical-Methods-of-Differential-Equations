\documentclass{article}


%调用
\usepackage{ctex}
\usepackage{xspace}
\usepackage{SIunits}%单位符号
\usepackage{xymtex}%化学符号
\usepackage{pst-optic}%光学符号



%自定义命令
\newcommand*{\myTestTimes}{1\xspace}
%\typein[\myTestTimes]{这是第几次测试?}
\newcommand*{\myName}{桑明达\xspace}
\newcommand*{\myNumber}{15300180062\xspace}
\newcommand*{\myHomeworkNumber}{第一周作业\xspace}
\newcommand*{\myArticleName}{微分方程数值解法\xspace}

\newcommand*{\myseries}[2][n]{\ensuremath{#2_1,#2_2,\dots,#2_#1}}


%制作页眉页脚
\usepackage{fancyhdr}
\pagestyle{fancy}
\lhead{\myHomeworkNumber}
\chead{\myArticleName}
\rhead{\myName \myNumber}
\lfoot{}
\cfoot{\thepage}
\rfoot{}
\renewcommand{\headrulewidth}{0.4pt}
\renewcommand{\footrulewidth}{0.4pt}

%标题
\title{\heiti \myArticleName \\ [2ex] \begin{large} \myHomeworkNumber \end{large}}
\author{\kaishu \myName \myNumber}
\date{\today}

% 正文区
\begin{document}
\maketitle

\part{第一部}

\chapter{第一章}

  hello!

  你好!\LaTeX

这是第\myTestTimes 次测试
  \newpage
  \@newpage

\section{第一节}


  数列可以写作\myseries{x},\myseries[m]{y}

  破折号:北京\raise{1.5pt}{------}上海

\section{第一小节}

\paragraph{第一段}

  米\metre,千克\kilogram,秒\second,摄氏度\degreecelsius,分钟\minute,\dots

  \bzdrh[pa]{4D==O;1D==CH$_{3}$SO$_{2}$2--N;3==CH$_{3}$}

  \lens[lensType=PDVG,focus=-2,spotAi=270,spotBi=90]

  \noindent \makebox[0pt][r]{\fbox{注意}}\qquad 这一行最前面有个宽度为0pt的盒子,并且没有首行缩进。

\newpage

\subparagraph{第一小段}

\begin{equation}
  1=1
\end{equation}

$$a=b\makebox{~(勾股定理)}$$
$$a=b\makebox[0pt][l]{~(勾股定理)}$$\\

\parindent=0pt
\makebox[60mm][s]{均匀分布这一行}\\
\makebox[60mm][s]{junyunfenbu}\\
\makebox[60mm][s]{j u n y u n f e n b u}

\begin{quote}
    名言
\end{quote}

这里插入一个脚注
\footnote{自己注释的脚注}

这里有一个交叉引用\label{text:1},详情参见第\pageref{text:1}页第\ref{text:1}节的介绍。\\

{\centering {\large 加大居中对齐标题}\\[4mm]
换行时加大4mm\\[1mm]
加大1mm\\}

\begin{verbatim}
这里是抄录格式,输出的文档会和我的输入一模一样
换行位置一样  空格长度一样 1  2   3    4个空格
NUM
\
$
%
7&*@#

\end{verbatim}

从1开始的以10为公差的等差数列,列出3项。
\multido{\n=1+10}{3}{\n,}
\newpage
\section{一些表格}

\begin{tabular}{|l|c|r|}
\hline
\multicolumn{3}{|c|}{表格示例}\\
\hline
col head &col head & col head\\
\hline
Leftv & col head & right\\
\cline{1-2}
aligned & items & aligned\\
\cline{2-3}
items&items&items\\\cline{1-2}

Left items& centered & right\\
\hline

\end{tabular}



\par
\begin{tabular}{c r @{.} l}
  \hline
  太阳系中的行星&\multicolumn{2}{c}{赤道半径km}\\
  \hline
  水星&2&44\\
  金星&6&1\\
  地球&6\:378&142\\
  \hline
\end{tabular}




\par
\begin{equation}
  \boxed{\int^{b}_{a}f(x)\,dx=-\int_{b}^{a}f(x)\,dx}
\end{equation}








\end{document}
