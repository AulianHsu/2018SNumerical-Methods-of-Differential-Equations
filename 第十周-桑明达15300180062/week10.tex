\documentclass{article}%book,report,letter

\usepackage{ctex}
\usepackage{fontspec}
%\usepackage{color}
%\usepackage{graphicx} %use graph format
%\usepackage{subfigure}
%\usepackage{epstopdf} %eps图片
\usepackage{amsmath}  %字体加粗
%\usepackage{math}
\usepackage{amsthm}
\usepackage{amssymb} %因为所以符号
%\usepackage{caption}
%\captionsetup[table]{labelsep=space}
\usepackage{float}%图片位置

%自定义命令
\newcommand*{\myTestTimes}{1\xspace}
%\typein[\myTestTimes]{这是第几次测试?}
\newcommand*{\myName}{桑明达\xspace}
\newcommand*{\myNumber}{15300180062\xspace}
\newcommand*{\myHomeworkNumber}{第十周作业\xspace}
\newcommand*{\myArticleName}{微分方程数值解法\xspace}

\newcommand*{\myseries}[2][n]{\ensuremath{#2_1,#2_2,\dots,#2_#1}}


%制作页眉页脚
\usepackage{fancyhdr}
\pagestyle{fancy}
\lhead{\myHomeworkNumber}
\chead{\myArticleName}
\rhead{\myName \myNumber}
\lfoot{}
\cfoot{\thepage}
\rfoot{}
\renewcommand{\headrulewidth}{0.4pt}
\renewcommand{\footrulewidth}{0.4pt}

%标题
\title{\heiti \myArticleName \\ [2ex] \begin{large} \myHomeworkNumber \end{large}}
\author{\kaishu \myName \myNumber}
\date{\today}

% 正文区
\begin{document}
\maketitle

%\newpage

\section{P147 Poincar\'{e}不等式离散形式(式3.1.54)}

\begin{proof}
\begin{align*}
	\mathbf{e}=&\left ( e_1,e_2,\dots ,e_N \right )\\
	\delta ^+_x \mathbf{e}=&\left ( \frac{e_2-e_1}{h},\frac{e_3-e_2}{h},\dots ,\frac{e_N-e_{N-1}}{h} \right )
\end{align*}

\begin{align*}
\therefore \sum_{i=1}^{N}e^2_i = & \sum_{i=1}^{N}\left (\sum_{k=0}^{i-1}\left (e_{k+1}-e_k  \right )  \right )^2 \\
\leq & \sum_{i=1}^{N}\left (\sum_{k=0}^{i-1}\left (e_{k+1}-e_k  \right )^2  \right ) \left (\sum_{k=0}^{i-1} 1^2  \right )  \\
\leq & N \sum_{i=1}^{N}\left (\sum_{k=0}^{i-1}\left (e_{k+1}-e_k  \right )^2  \right ) \\
\leq & N \sum_{i=1}^{N}\left (\sum_{k=0}^{N-1}\left (e_{k+1}-e_k  \right )^2  \right ) \\
= &  N^2 \sum_{k=0}^{N-1}\left (e_{k+1}-e_k  \right )^2   \\
= &   \sum_{k=0}^{N-1} \left ( \frac{e_{k+1}-e_k}{h} \right )^2  \\
\therefore \left \| \mathbf{e}  \right \|_{\mathit{l}^2} \leq & \left \| \delta ^+_x \mathbf{e}  \right \|_{\mathit{l}^2}
\end{align*}
\end{proof}

\section{P151 3 三点差分格式极值原理和最大模误差估计}
\begin{proof}
	(1)由三点差分格式有,
	$$-f_i=\frac{d}{dx}\left ( a(x)\frac{du}{dx} \right ) \bigg |_{x=x_i} \approx \frac{1}{h}\left ( a(x_{i+\frac{1}{2}})\frac{u_{i+1}-u_i}{h}-a(x_{i-\frac{1}{2}})\frac{u_{i}-u_{i-1}}{h} \right )$$

上式可以化为
$$ a(x_{i+\frac{1}{2}})\left (u_{i+1}-u_i  \right ) =-h^2f_i+a(x_{i-\frac{1}{2}})\left (u_{i}-u_{i-1}  \right )$$

当$f_i \geq 0$时,如果$u_i$取得最小值,因为$a(x) \geq \alpha > 0$,所以上式左边大于等于0,右边小于等于0。得到i=0或n,或者得到$u_i$恒为常数,即u(x)最小值只能在边界达到。

当$f_i \leq 0$时,如果$u_i$取得最大值,因为$a(x) \geq \alpha > 0$,所以上式左边小于等于0,右边大于等于0。得到i=0或n,或者得到$u_i$恒为常数,即u(x)最大值只能在边界达到。

极值原理得证。

(2)记\begin{align*}
\mathbf{u}=& \left ( \myseries[N-1]{u} \right )^T\\
\mathbf{e}=& \left ( \myseries[N-1]{e} \right )^T\\
\mathbf{f}=& \left ( \myseries[N-1]{f} \right )^T\\
\mathbf{R}=& \left ( \myseries[N-1]{R} \right )^T\\
\end{align*}
$\mathbf{A}$=\begin{pmatrix}
 a_{\frac{3}{2}}+a_{\frac{1}{2}}&-a_{\frac{3}{2}}  &  &  & \\
 -a_{\frac{3}{2}}& a_{\frac{5}{2}}+a_{\frac{3}{2}} & -a_{\frac{5}{2}} &  & \\
 & -a_{\frac{5}{2}} & a_{\frac{7}{2}}+a_{\frac{5}{2}} & \ddots & \\
 &  & \ddots & \ddots & -a_{N-2+\frac{1}{2}} \\
 &  &  & -a_{N-1-\frac{1}{2}} & a_{N-1-\frac{1}{2}}+a_{N-1+\frac{1}{2}}
\end{pmatrix}

那么有
\begin{align*}
	\mathbf{Au}=& h^2 \mathbf{f}\\
	\mathbf{Ae}=& h^2 \mathbf{R}\\
\end{align*}
因为$a(x) \geq \alpha > 0$,所以$ \mathbf{A}$是正定阵,所以
\begin{align*}
	\mathbf{e}=& h^2 \mathbf{{A}^{-1}R}\\
	\left \| \mathbf{e}  \right \|_{\mathit{l}^{\infty }} =& h^2 \left \| \mathbf{{A}^{-1}R} \right \|_{\mathit{l}^{\infty }} \\
  \leq & h^2 \left \| \mathbf{{A}^{-1}} \right \| \left \| \mathbf{R} \right \| _{\mathit{l}^{\infty }}
\end{align*}
\end{proof}
\section{P164 1 五点差分格式问题}
\begin{proof}
(1)$\mathbf{A}$=\begin{pmatrix}
\mathbf{S} & \mathbf{T} &  & \\
\mathbf{T} & \ddots & \ddots & \\
 & \ddots & \ddots & \mathbf{T}\\
 &  & \mathbf{T} & \mathbf{S}
\end{pmatrix}

其中,
\begin{align*}
	\mathbf{T}=& -\frac{1}{h^2} \mathbf{I}\\
	\mathbf{S}=& \frac{1}{h^2} \begin{pmatrix}
	4 & -1 &  & \\
	-1 & \ddots & \ddots & \\
	 & \ddots & \ddots & -1\\
	 &  & -1 & 4
	\end{pmatrix}\\
\end{align*}
所以,$\mathbf{A}+\mathbf{I}$=\begin{pmatrix}
\mathbf{S}+\mathbf{I} & \mathbf{T} &  & \\
\mathbf{T} & \ddots & \ddots & \\
 & \ddots & \ddots & \mathbf{T}\\
 &  & \mathbf{T} & \mathbf{S}+\mathbf{I}
\end{pmatrix}

其中,
\begin{align*}
	\mathbf{S}+\mathbf{I}=& \begin{pmatrix}
	\frac{4}{h^2}+1 & -\frac{1}{h^2} &  & \\
	-\frac{1}{h} & \ddots & \ddots & \\
	 & \ddots & \ddots & -\frac{1}{h}\\
	 &  & -\frac{1}{h^2} & \frac{4}{h^2}+1
	\end{pmatrix}
\end{align*}

容易得到$\mathbf{A}+\mathbf{I}$也是成块TST矩阵。

(2)记iii

(3)



\end{proof}
\end{document}
